\section{Portfolio: The Mechanists (Mandate for Corporeal Perfection via the Abandonment of Flesh)}
Ships overview for all groups (Work In Progress): \href{http://vegastrike.sourceforge.net/wiki/Artstyle\_guide:Overview\_Guide}{Ship Overview} \\
Art-style Guide (Work In Progress): \href{http://vegastrike.sourceforge.net/wiki/Artstyle\_guide:Mechanist}{Artstyle Guide} \\
Species overview: \href{http://vegastrike.sourceforge.net/wiki/Species:Humanity}{Species:Humanity} \\

\subsection{Origin}
\begin{itemize}
\item Sector: 
\item System: 
\item Origin Planet:  
\item Gravity: 

\item Atmosphere: 

\item Primary liquid bodies: 

\item Average temperature of homeworld (pre-industrialization):

\item Sun: 

\item Primary challenges (pre-industrialization): 
\end{itemize}

%ORIGIN COMMENTS GO HERE

\subsubsection{Habitat}

\subsection{Physical}
\begin{itemize}
\item Dimensions: 

\item Mass: 

\item Skeletal system: 

\item Major divisions: 

\item Senses: 

\item Visual acuity: 

\item Chemosense: 

\item Locomotion: 

\item Manipulators: 

\item Textural appearance: 
\end{itemize}

%PHYSICAL COMMENTS GO HERE

\subsection{Mental}


\subsection{Technological}
\begin{itemize}
\item Tech: 


\item Weapons:

\item Tactics:
\begin{itemize}
\item    Small groups: 
\item    Large groups/Fleets: 
\end{itemize}


\item Installations:


\end{itemize}

\subsection{Culture}

\subsubsection{Factions and Organizational Groups}
Listed below are noteworthy Aeran sub-factions and organizational groups: 
\begin{itemize}
\item FACTIONS GO HERE
\end{itemize}

\subsubsection{Religion}

\subsubsection{Cultural Aesthetics}

\subsection{Writing, numbers, and insignia}

\subsection{Faction: PRIMARY FACTION}
%Faction data 
%Aera 
%Species 	Aera 
%Homeworld (Origin) 	Aeneth 
%Capital 	Aeneth 


\subsubsection{A Brief History of the PRIMARY FACTION}


\subsubsection{Development}

\subsubsection{Culture}

\subsubsection{Organization}

\subsection{Faction: OTHER FACTIONS}

%Faction data 
%Merchant Marines 
%Species 	Aera 
%Homeworld (Origin) 	Aeneth 
%Capital 	Aeneth 


\subsection{Vessels Style}

\subsubsection{Style Overview}
\begin{itemize}

\item Primary distinguishing color ranges: 

\item Common accent colors:

\item Primary lighting color:

\item Frequently visible: 

\item Rarely visible:

\item Seen inside, but not out: 

\item Moving parts(non-turret): 

\item Capital vs. light craft: 

\end{itemize}

\subsubsection{Surface features of large vessels}

\subsubsection{Small things found on the hull of a large  vessel}
\begin{itemize}
\item Service/Maintenance hatches
\end{itemize}
{\it Somewhat larger things found on the hull of a large Aeran vessel:}
\begin{itemize}
\item Escape pod launcher ports
\end{itemize}
{\it Yet larger things ... :}
\begin{itemize}
\item Pinnace/lander launch bay (non-carrier vessels)
\end{itemize}

\subsection{Listing of vessels}

Existing concept art is not particularly canonical. Please redesign.

\subsubsection{Interceptor Mk. 273 (Mechanist Interceptor)}

    The central most part of the hull is about three times as long as it is high, and twice as high as it is wide. It is roughly oblong, in frontal cross section, and somewhat more ovoid in side cross section. There is \_no\_ visible cockpit as such, because NO MECHANIST CRAFT HAVE VISIBLE COCKPIT AREAS . Both pilot and co-pilot are ensconced in the centermost portion of the vessel, having been loaded through hatches on the fore and aft undersides. Both the frontal and rearmost parts of the central hull are bedecked with sensory and transmission equipment. On each side of the central hull are slightly dimpled extensions, squarish (though rounded in a concave fashion, hence the dimpling) about as wide again as the central hull region, and slightly extended beyond both its top and bottom. Together with the central hull, these form what would be, if the central hull were cleaved flush at front and back, a cube dimpled on sides and top, not unlike a block of chalk used on four sides to coat the tips of pool/billiards cues, although here, the outside edges are rounded as well. Situated in each of these dimples is a turreted engine fixture. Those on the top are larger than those on the sides, proportionately to the ratio of the width to the height of the "dimpled block". Each engine fixture is consists of two main propulsory engines, opening rearward only, incorporated into a round-edged block of thickness commensurable with the width of the centermost hull region (rounded also at the corners from the top-down sense). Thus, the top and bottom engine blocks have connective regions somewhat less wide than the centermost hull region, and engines of diameter similar to the width of the centermost hull region (a bit on the smaller than side of similar), and the ones on the side have the same proportions, but are smaller. Atop each turreted engine mount is a rocket pod full of small, rapid launch dumb projectiles.

Embedded in the each side block is a very large spinal mounted gun, running 80\% of the length of the craft, and centered lengthwise with respect to the rest of the vessel. The maximum diameter of the central tube describing each gun is ~1/2 the width of the central section of the craft. They are not merely open tubes - any opening depicted should be quite small in comparison to the diameter. Various accessories to the gun emplacements (capacitors, cooling systems) are mounted above and below the gun, but not to the sides.

For the sake of scaling details, assume the width of the central section to be ~5 meters, thus making the entire vessel ~30 meters long, and ~30 meters wide at the engines (though only for ~8 meters at the center of the length of the vessel (+-4m), 15 m wide otherwise) and a height of 30 meters at the engines (but only for 12 meters (+-6m from center), 10m high otherwise. 

\subsubsection{Interceptor Mk. 212 (Mechanist Interceptor)}

The main hull is an irregular hexagonal prism, with alternating sides of length X and 2X, a length X side "down". This configuration dominates half the length of the vessel. It is extended half again on both front and back (the hexagonal faces being front/back) with a regular hexagonal prism of side length X, centered, flat side "down". Each extension is subdivided into 24 equilateral triangular regions, each of which houses a (not particularly triangular) missile launch tube.Three non-descript spinal mounts (i.e. make the support structure for the guns, not the guns, those will be added later in modular fashion, unlike the above integrated weapons) begin 1/3 of the way forward on the main hull on length X sides, and continue on past the end of the main hull, running over the forward extension. Underneath and around the gun mounts, alongside the extension, are sensors and transceivers, providing some degree of merging of the extension and the main hull. Alongside the rear extension, there are also sensors and transceivers, but less so, as the more striking features are the three rows of decoy launchers, one row aligned with each of the X length regions of the main hull. On each of the 2X sides of the main hull, there are turreted engine emplacements, similar to those on the Mk 273, excepting that instead of rocket pods atop the two larger engines there are two slightly smaller engines that abut the larger two, each being "above" and towards the other engine, resulting in a sort of a "haircurler" arrangement of engines upon the turret mount. Assume X to be ~ >=2 meters, for a total width of each engine turret of ~8 meters, and assume the main hull region to be ~16 meters long, with each extension then being ~8 meters long. There is a single hatch for pilot insertion on the top front of the main hull segment.

The main geometry described here is simple - so spend lots of polygons on the connective structures attaching the extensions to the main hull, and all of the sensors/transcievers, decoy launchers, gun support structure, engine turrets, and missile launchers. Done properly, the appearance should be no more low-poly than an any object, such as a tank, with an intrinsically angular core presence. Indeed, when I say "irregular hexagonal prism" this doesn't impose a limit of no more than six large quads for the center part of the ship- this is an indication that the fundamental shape is that of an irregular hexagonal prism - precise details beyond that are left for the artist to manipulate - if I've mentioned it, I thought it was important, if I didn't mention something, then use your judgment as to whether it's appropriate to add - the description isn't meant to be exhaustive.




% LocalWords:  Aerans Aeran Artstyle Aera Pinnace Acrotatus Agasicles Agesilaus
% LocalWords:  Agesipolis Agis Alcmenes Anaxander Wiki Anaxandridas Rlaan Areus
% LocalWords:  Anaxidamus Ariston Charillus Cleombrotus Cleomenes Demaratus Uln
% LocalWords:  Theopompus Dorissus Echestratus Eurycratides Eurypon Leons Bzbr
% LocalWords:  Nicander Pausanias Pleistarchus UnAeranned Pleistoanax Andolian
% LocalWords:  Spaceborn's MacGyver Polydectes Polydorus EVAs Procles Prytanis
% LocalWords:  Soos Teleclus terraforming Aenethforming Zeuxidamus homeworld
% LocalWords:  coreward chemoreception Klk'k Aeneth ecologies Shmrn
